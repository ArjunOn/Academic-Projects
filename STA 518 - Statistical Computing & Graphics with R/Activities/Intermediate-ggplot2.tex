% Options for packages loaded elsewhere
\PassOptionsToPackage{unicode}{hyperref}
\PassOptionsToPackage{hyphens}{url}
%
\documentclass[
]{article}
\usepackage{amsmath,amssymb}
\usepackage{iftex}
\ifPDFTeX
  \usepackage[T1]{fontenc}
  \usepackage[utf8]{inputenc}
  \usepackage{textcomp} % provide euro and other symbols
\else % if luatex or xetex
  \usepackage{unicode-math} % this also loads fontspec
  \defaultfontfeatures{Scale=MatchLowercase}
  \defaultfontfeatures[\rmfamily]{Ligatures=TeX,Scale=1}
\fi
\usepackage{lmodern}
\ifPDFTeX\else
  % xetex/luatex font selection
\fi
% Use upquote if available, for straight quotes in verbatim environments
\IfFileExists{upquote.sty}{\usepackage{upquote}}{}
\IfFileExists{microtype.sty}{% use microtype if available
  \usepackage[]{microtype}
  \UseMicrotypeSet[protrusion]{basicmath} % disable protrusion for tt fonts
}{}
\makeatletter
\@ifundefined{KOMAClassName}{% if non-KOMA class
  \IfFileExists{parskip.sty}{%
    \usepackage{parskip}
  }{% else
    \setlength{\parindent}{0pt}
    \setlength{\parskip}{6pt plus 2pt minus 1pt}}
}{% if KOMA class
  \KOMAoptions{parskip=half}}
\makeatother
\usepackage{xcolor}
\usepackage[margin=1in]{geometry}
\usepackage{color}
\usepackage{fancyvrb}
\newcommand{\VerbBar}{|}
\newcommand{\VERB}{\Verb[commandchars=\\\{\}]}
\DefineVerbatimEnvironment{Highlighting}{Verbatim}{commandchars=\\\{\}}
% Add ',fontsize=\small' for more characters per line
\usepackage{framed}
\definecolor{shadecolor}{RGB}{248,248,248}
\newenvironment{Shaded}{\begin{snugshade}}{\end{snugshade}}
\newcommand{\AlertTok}[1]{\textcolor[rgb]{0.94,0.16,0.16}{#1}}
\newcommand{\AnnotationTok}[1]{\textcolor[rgb]{0.56,0.35,0.01}{\textbf{\textit{#1}}}}
\newcommand{\AttributeTok}[1]{\textcolor[rgb]{0.13,0.29,0.53}{#1}}
\newcommand{\BaseNTok}[1]{\textcolor[rgb]{0.00,0.00,0.81}{#1}}
\newcommand{\BuiltInTok}[1]{#1}
\newcommand{\CharTok}[1]{\textcolor[rgb]{0.31,0.60,0.02}{#1}}
\newcommand{\CommentTok}[1]{\textcolor[rgb]{0.56,0.35,0.01}{\textit{#1}}}
\newcommand{\CommentVarTok}[1]{\textcolor[rgb]{0.56,0.35,0.01}{\textbf{\textit{#1}}}}
\newcommand{\ConstantTok}[1]{\textcolor[rgb]{0.56,0.35,0.01}{#1}}
\newcommand{\ControlFlowTok}[1]{\textcolor[rgb]{0.13,0.29,0.53}{\textbf{#1}}}
\newcommand{\DataTypeTok}[1]{\textcolor[rgb]{0.13,0.29,0.53}{#1}}
\newcommand{\DecValTok}[1]{\textcolor[rgb]{0.00,0.00,0.81}{#1}}
\newcommand{\DocumentationTok}[1]{\textcolor[rgb]{0.56,0.35,0.01}{\textbf{\textit{#1}}}}
\newcommand{\ErrorTok}[1]{\textcolor[rgb]{0.64,0.00,0.00}{\textbf{#1}}}
\newcommand{\ExtensionTok}[1]{#1}
\newcommand{\FloatTok}[1]{\textcolor[rgb]{0.00,0.00,0.81}{#1}}
\newcommand{\FunctionTok}[1]{\textcolor[rgb]{0.13,0.29,0.53}{\textbf{#1}}}
\newcommand{\ImportTok}[1]{#1}
\newcommand{\InformationTok}[1]{\textcolor[rgb]{0.56,0.35,0.01}{\textbf{\textit{#1}}}}
\newcommand{\KeywordTok}[1]{\textcolor[rgb]{0.13,0.29,0.53}{\textbf{#1}}}
\newcommand{\NormalTok}[1]{#1}
\newcommand{\OperatorTok}[1]{\textcolor[rgb]{0.81,0.36,0.00}{\textbf{#1}}}
\newcommand{\OtherTok}[1]{\textcolor[rgb]{0.56,0.35,0.01}{#1}}
\newcommand{\PreprocessorTok}[1]{\textcolor[rgb]{0.56,0.35,0.01}{\textit{#1}}}
\newcommand{\RegionMarkerTok}[1]{#1}
\newcommand{\SpecialCharTok}[1]{\textcolor[rgb]{0.81,0.36,0.00}{\textbf{#1}}}
\newcommand{\SpecialStringTok}[1]{\textcolor[rgb]{0.31,0.60,0.02}{#1}}
\newcommand{\StringTok}[1]{\textcolor[rgb]{0.31,0.60,0.02}{#1}}
\newcommand{\VariableTok}[1]{\textcolor[rgb]{0.00,0.00,0.00}{#1}}
\newcommand{\VerbatimStringTok}[1]{\textcolor[rgb]{0.31,0.60,0.02}{#1}}
\newcommand{\WarningTok}[1]{\textcolor[rgb]{0.56,0.35,0.01}{\textbf{\textit{#1}}}}
\usepackage{longtable,booktabs,array}
\usepackage{calc} % for calculating minipage widths
% Correct order of tables after \paragraph or \subparagraph
\usepackage{etoolbox}
\makeatletter
\patchcmd\longtable{\par}{\if@noskipsec\mbox{}\fi\par}{}{}
\makeatother
% Allow footnotes in longtable head/foot
\IfFileExists{footnotehyper.sty}{\usepackage{footnotehyper}}{\usepackage{footnote}}
\makesavenoteenv{longtable}
\usepackage{graphicx}
\makeatletter
\def\maxwidth{\ifdim\Gin@nat@width>\linewidth\linewidth\else\Gin@nat@width\fi}
\def\maxheight{\ifdim\Gin@nat@height>\textheight\textheight\else\Gin@nat@height\fi}
\makeatother
% Scale images if necessary, so that they will not overflow the page
% margins by default, and it is still possible to overwrite the defaults
% using explicit options in \includegraphics[width, height, ...]{}
\setkeys{Gin}{width=\maxwidth,height=\maxheight,keepaspectratio}
% Set default figure placement to htbp
\makeatletter
\def\fps@figure{htbp}
\makeatother
\setlength{\emergencystretch}{3em} % prevent overfull lines
\providecommand{\tightlist}{%
  \setlength{\itemsep}{0pt}\setlength{\parskip}{0pt}}
\setcounter{secnumdepth}{-\maxdimen} % remove section numbering
\ifLuaTeX
  \usepackage{selnolig}  % disable illegal ligatures
\fi
\IfFileExists{bookmark.sty}{\usepackage{bookmark}}{\usepackage{hyperref}}
\IfFileExists{xurl.sty}{\usepackage{xurl}}{} % add URL line breaks if available
\urlstyle{same}
\hypersetup{
  pdftitle={Intermediate-ggplot2},
  pdfauthor={Arjun Mannem},
  hidelinks,
  pdfcreator={LaTeX via pandoc}}

\title{Intermediate-ggplot2}
\author{Arjun Mannem}
\date{2024-02-03}

\begin{document}
\maketitle

\begin{Shaded}
\begin{Highlighting}[]
\FunctionTok{library}\NormalTok{(riskCommunicator)}
\FunctionTok{library}\NormalTok{(tidyverse)}
\FunctionTok{library}\NormalTok{(skimr)}
\FunctionTok{library}\NormalTok{(knitr)}
\FunctionTok{library}\NormalTok{(ggthemes)}
\FunctionTok{library}\NormalTok{(ggpubr)}
\FunctionTok{library}\NormalTok{(patchwork)}
\end{Highlighting}
\end{Shaded}

First, let's load the FHS data set from the riskCommunicator package
using the code below. Then, submit ?framingham to the Console to view
the help file for the Framingham data set in RStudio.

\begin{Shaded}
\begin{Highlighting}[]
\FunctionTok{data}\NormalTok{(framingham, }\AttributeTok{package =} \StringTok{"riskCommunicator"}\NormalTok{)}
\end{Highlighting}
\end{Shaded}

Select the first 10 variables from the Framingham dataset and store it
as a new data frame called framinghamSub using the select() function.
Also, update the SEX variable to have the values ``Male'' and ``Female''
rather than 1 and 2, and the CURSMOKE variable to have the values
``Yes'' and ``No'' rather than 1 and 0 using the mutate() and
case\_when() functions. This should be your new dataset to be used for
the rest of the assignment.

\begin{Shaded}
\begin{Highlighting}[]
\NormalTok{framinghamSub }\OtherTok{\textless{}{-}}\NormalTok{ framingham }\SpecialCharTok{|\textgreater{}}
\NormalTok{  dplyr}\SpecialCharTok{::}\FunctionTok{select}\NormalTok{(}\DecValTok{1}\SpecialCharTok{:}\DecValTok{10}\NormalTok{) }\SpecialCharTok{|\textgreater{}}
  \FunctionTok{mutate}\NormalTok{(}\AttributeTok{SexNew =} \FunctionTok{case\_when}\NormalTok{(SEX }\SpecialCharTok{==} \DecValTok{1} \SpecialCharTok{\textasciitilde{}} \StringTok{"Male"}\NormalTok{,}
\NormalTok{                            SEX }\SpecialCharTok{==} \DecValTok{2} \SpecialCharTok{\textasciitilde{}} \StringTok{"Female"}\NormalTok{,}
                            \ConstantTok{TRUE} \SpecialCharTok{\textasciitilde{}} \StringTok{"Other"}\NormalTok{),}
         \AttributeTok{CurSmokeNew =} \FunctionTok{case\_when}\NormalTok{(CURSMOKE }\SpecialCharTok{==} \DecValTok{0} \SpecialCharTok{\textasciitilde{}} \StringTok{"No"}\NormalTok{,}
\NormalTok{                            CURSMOKE }\SpecialCharTok{==} \DecValTok{1} \SpecialCharTok{\textasciitilde{}} \StringTok{"Yes"}\NormalTok{,}
                            \ConstantTok{TRUE} \SpecialCharTok{\textasciitilde{}} \StringTok{"Other"}\NormalTok{))}
\end{Highlighting}
\end{Shaded}

Use the skim() function from the skimr package to explore other
characteristics of the subset of the data.

\begin{Shaded}
\begin{Highlighting}[]
\FunctionTok{library}\NormalTok{(skimr)}

\FunctionTok{skim}\NormalTok{(framinghamSub)}
\end{Highlighting}
\end{Shaded}

\begin{longtable}[]{@{}ll@{}}
\caption{Data summary}\tabularnewline
\toprule\noalign{}
\endfirsthead
\endhead
\bottomrule\noalign{}
\endlastfoot
Name & framinghamSub \\
Number of rows & 11627 \\
Number of columns & 12 \\
\_\_\_\_\_\_\_\_\_\_\_\_\_\_\_\_\_\_\_\_\_\_\_ & \\
Column type frequency: & \\
character & 2 \\
numeric & 10 \\
\_\_\_\_\_\_\_\_\_\_\_\_\_\_\_\_\_\_\_\_\_\_\_\_ & \\
Group variables & None \\
\end{longtable}

\textbf{Variable type: character}

\begin{longtable}[]{@{}
  >{\raggedright\arraybackslash}p{(\columnwidth - 14\tabcolsep) * \real{0.1944}}
  >{\raggedleft\arraybackslash}p{(\columnwidth - 14\tabcolsep) * \real{0.1389}}
  >{\raggedleft\arraybackslash}p{(\columnwidth - 14\tabcolsep) * \real{0.1944}}
  >{\raggedleft\arraybackslash}p{(\columnwidth - 14\tabcolsep) * \real{0.0556}}
  >{\raggedleft\arraybackslash}p{(\columnwidth - 14\tabcolsep) * \real{0.0556}}
  >{\raggedleft\arraybackslash}p{(\columnwidth - 14\tabcolsep) * \real{0.0833}}
  >{\raggedleft\arraybackslash}p{(\columnwidth - 14\tabcolsep) * \real{0.1250}}
  >{\raggedleft\arraybackslash}p{(\columnwidth - 14\tabcolsep) * \real{0.1528}}@{}}
\toprule\noalign{}
\begin{minipage}[b]{\linewidth}\raggedright
skim\_variable
\end{minipage} & \begin{minipage}[b]{\linewidth}\raggedleft
n\_missing
\end{minipage} & \begin{minipage}[b]{\linewidth}\raggedleft
complete\_rate
\end{minipage} & \begin{minipage}[b]{\linewidth}\raggedleft
min
\end{minipage} & \begin{minipage}[b]{\linewidth}\raggedleft
max
\end{minipage} & \begin{minipage}[b]{\linewidth}\raggedleft
empty
\end{minipage} & \begin{minipage}[b]{\linewidth}\raggedleft
n\_unique
\end{minipage} & \begin{minipage}[b]{\linewidth}\raggedleft
whitespace
\end{minipage} \\
\midrule\noalign{}
\endhead
\bottomrule\noalign{}
\endlastfoot
SexNew & 0 & 1 & 4 & 6 & 0 & 2 & 0 \\
CurSmokeNew & 0 & 1 & 2 & 3 & 0 & 2 & 0 \\
\end{longtable}

\textbf{Variable type: numeric}

\begin{longtable}[]{@{}
  >{\raggedright\arraybackslash}p{(\columnwidth - 20\tabcolsep) * \real{0.1197}}
  >{\raggedleft\arraybackslash}p{(\columnwidth - 20\tabcolsep) * \real{0.0855}}
  >{\raggedleft\arraybackslash}p{(\columnwidth - 20\tabcolsep) * \real{0.1197}}
  >{\raggedleft\arraybackslash}p{(\columnwidth - 20\tabcolsep) * \real{0.0940}}
  >{\raggedleft\arraybackslash}p{(\columnwidth - 20\tabcolsep) * \real{0.0940}}
  >{\raggedleft\arraybackslash}p{(\columnwidth - 20\tabcolsep) * \real{0.0684}}
  >{\raggedleft\arraybackslash}p{(\columnwidth - 20\tabcolsep) * \real{0.0940}}
  >{\raggedleft\arraybackslash}p{(\columnwidth - 20\tabcolsep) * \real{0.0940}}
  >{\raggedleft\arraybackslash}p{(\columnwidth - 20\tabcolsep) * \real{0.0940}}
  >{\raggedleft\arraybackslash}p{(\columnwidth - 20\tabcolsep) * \real{0.0855}}
  >{\raggedright\arraybackslash}p{(\columnwidth - 20\tabcolsep) * \real{0.0513}}@{}}
\toprule\noalign{}
\begin{minipage}[b]{\linewidth}\raggedright
skim\_variable
\end{minipage} & \begin{minipage}[b]{\linewidth}\raggedleft
n\_missing
\end{minipage} & \begin{minipage}[b]{\linewidth}\raggedleft
complete\_rate
\end{minipage} & \begin{minipage}[b]{\linewidth}\raggedleft
mean
\end{minipage} & \begin{minipage}[b]{\linewidth}\raggedleft
sd
\end{minipage} & \begin{minipage}[b]{\linewidth}\raggedleft
p0
\end{minipage} & \begin{minipage}[b]{\linewidth}\raggedleft
p25
\end{minipage} & \begin{minipage}[b]{\linewidth}\raggedleft
p50
\end{minipage} & \begin{minipage}[b]{\linewidth}\raggedleft
p75
\end{minipage} & \begin{minipage}[b]{\linewidth}\raggedleft
p100
\end{minipage} & \begin{minipage}[b]{\linewidth}\raggedright
hist
\end{minipage} \\
\midrule\noalign{}
\endhead
\bottomrule\noalign{}
\endlastfoot
RANDID & 0 & 1.00 & 5004740.92 & 2900877.44 & 2448.00 & 2474378.00 &
5006008.00 & 7472730.00 & 9999312.0 & ▇▇▇▇▇ \\
SEX & 0 & 1.00 & 1.57 & 0.50 & 1.00 & 1.00 & 2.00 & 2.00 & 2.0 &
▆▁▁▁▇ \\
TOTCHOL & 409 & 0.96 & 241.16 & 45.37 & 107.00 & 210.00 & 238.00 &
268.00 & 696.0 & ▅▇▁▁▁ \\
AGE & 0 & 1.00 & 54.79 & 9.56 & 32.00 & 48.00 & 54.00 & 62.00 & 81.0 &
▂▇▇▅▁ \\
SYSBP & 0 & 1.00 & 136.32 & 22.80 & 83.50 & 120.00 & 132.00 & 149.00 &
295.0 & ▆▇▁▁▁ \\
DIABP & 0 & 1.00 & 83.04 & 11.66 & 30.00 & 75.00 & 82.00 & 90.00 & 150.0
& ▁▅▇▁▁ \\
CURSMOKE & 0 & 1.00 & 0.43 & 0.50 & 0.00 & 0.00 & 0.00 & 1.00 & 1.0 &
▇▁▁▁▆ \\
CIGPDAY & 79 & 0.99 & 8.25 & 12.19 & 0.00 & 0.00 & 0.00 & 20.00 & 90.0 &
▇▂▁▁▁ \\
BMI & 52 & 1.00 & 25.88 & 4.10 & 14.43 & 23.09 & 25.48 & 28.07 & 56.8 &
▃▇▁▁▁ \\
DIABETES & 0 & 1.00 & 0.05 & 0.21 & 0.00 & 0.00 & 0.00 & 0.00 & 1.0 &
▇▁▁▁▁ \\
\end{longtable}

Modify the code below to set a default ggplot theme for the entire
document to a complete theme of your choice from the ggplot2 package or
the ggthemes package.

\begin{Shaded}
\begin{Highlighting}[]
\FunctionTok{theme\_set}\NormalTok{(ggthemes}\SpecialCharTok{::}\FunctionTok{theme\_few}\NormalTok{())}
\end{Highlighting}
\end{Shaded}

Make a scatter plot between diastolic (DIABP) and systolic (SYSBP) blood
pressure with a ``facet'' by the sex of the participant (SEX).

\begin{Shaded}
\begin{Highlighting}[]
\NormalTok{framinghamSub }\SpecialCharTok{|\textgreater{}} 
  \FunctionTok{ggplot}\NormalTok{(}\FunctionTok{aes}\NormalTok{(}\AttributeTok{x =}\NormalTok{ SYSBP, }\AttributeTok{y =}\NormalTok{ DIABP)) }\SpecialCharTok{+}
  \FunctionTok{geom\_point}\NormalTok{() }\SpecialCharTok{+}
  \FunctionTok{facet\_grid}\NormalTok{(. }\SpecialCharTok{\textasciitilde{}}\NormalTok{ SexNew) }\CommentTok{\#Try  facet\_grid(CurSmokeNew \textasciitilde{} SexNew)}
\end{Highlighting}
\end{Shaded}

\includegraphics{Intermediate-ggplot2_files/figure-latex/unnamed-chunk-6-1.pdf}

Also manually set the alpha aesthetic to be 0.2.

\begin{Shaded}
\begin{Highlighting}[]
\NormalTok{framinghamSub }\SpecialCharTok{|\textgreater{}} 
  \FunctionTok{ggplot}\NormalTok{(}\FunctionTok{aes}\NormalTok{(}\AttributeTok{x =}\NormalTok{ SYSBP, }\AttributeTok{y =}\NormalTok{ DIABP)) }\SpecialCharTok{+}
  \FunctionTok{geom\_point}\NormalTok{(}\AttributeTok{alpha =} \FloatTok{0.2}\NormalTok{) }\SpecialCharTok{+}
  \FunctionTok{facet\_grid}\NormalTok{(CurSmokeNew }\SpecialCharTok{\textasciitilde{}}\NormalTok{ SexNew)}
\end{Highlighting}
\end{Shaded}

\includegraphics{Intermediate-ggplot2_files/figure-latex/unnamed-chunk-7-1.pdf}

Add a guides(color = FALSE) layer to suppress the legend since it is
redundant.

\begin{Shaded}
\begin{Highlighting}[]
\NormalTok{framinghamSub }\SpecialCharTok{|\textgreater{}}
  \FunctionTok{ggplot}\NormalTok{(}\FunctionTok{aes}\NormalTok{(}\AttributeTok{x =}\NormalTok{ SYSBP, }\AttributeTok{y =}\NormalTok{ DIABP,}
             \AttributeTok{size =}\NormalTok{ CIGPDAY,}
             \AttributeTok{color =}\NormalTok{ SexNew)) }\SpecialCharTok{+}
  \FunctionTok{geom\_point}\NormalTok{(}\AttributeTok{alpha =} \FloatTok{0.3}\NormalTok{) }\SpecialCharTok{+}
  \FunctionTok{facet\_grid}\NormalTok{(. }\SpecialCharTok{\textasciitilde{}}\NormalTok{ SexNew) }\SpecialCharTok{+}
  \FunctionTok{guides}\NormalTok{(}\AttributeTok{color =} \StringTok{"none"}\NormalTok{) }\SpecialCharTok{+}
  \FunctionTok{labs}\NormalTok{(}\AttributeTok{title =} \StringTok{"Diastolic by Systolic pressure"}\NormalTok{,}
       \AttributeTok{y =} \StringTok{"Diastolic BP (mmHg)"}\NormalTok{,}
       \AttributeTok{x =} \StringTok{"Systolic BP (mmHg)"}\NormalTok{)}
\end{Highlighting}
\end{Shaded}

\includegraphics{Intermediate-ggplot2_files/figure-latex/unnamed-chunk-8-1.pdf}

Also include the size of the data points as mapped by the number of
cigarettes smoked per day (CIGPDAY), add a color-blind friendly palette
for coloring the points, and position the legend at the bottom of the
plot.

\begin{Shaded}
\begin{Highlighting}[]
\NormalTok{framinghamSub }\SpecialCharTok{|\textgreater{}}
  \FunctionTok{ggplot}\NormalTok{(}\FunctionTok{aes}\NormalTok{(}\AttributeTok{x =}\NormalTok{ SYSBP, }\AttributeTok{y =}\NormalTok{ DIABP,}
             \AttributeTok{size =}\NormalTok{ CIGPDAY,}
             \AttributeTok{color =}\NormalTok{ SexNew)) }\SpecialCharTok{+}
  \FunctionTok{geom\_point}\NormalTok{(}\AttributeTok{alpha =} \FloatTok{0.3}\NormalTok{) }\SpecialCharTok{+}
  \FunctionTok{scale\_color\_colorblind}\NormalTok{() }\SpecialCharTok{+}
  \FunctionTok{facet\_grid}\NormalTok{(. }\SpecialCharTok{\textasciitilde{}}\NormalTok{ SexNew) }\SpecialCharTok{+}
  \FunctionTok{guides}\NormalTok{(}\AttributeTok{color =} \StringTok{"none"}\NormalTok{) }\SpecialCharTok{+}
  \FunctionTok{labs}\NormalTok{(}\AttributeTok{title =} \StringTok{"Diastolic by Systolic pressure"}\NormalTok{,}
       \AttributeTok{y =} \StringTok{"Diastolic BP (mmHg)"}\NormalTok{,}
       \AttributeTok{x =} \StringTok{"Systolic BP (mmHg)"}\NormalTok{,}
       \AttributeTok{caption =} \StringTok{"Data Source: Framington Heart Study, https://www.framinghamheartstudy.org/"}\NormalTok{,}
       \AttributeTok{size =} \StringTok{"Cigarettes Smoked daily"}\NormalTok{) }\SpecialCharTok{+}
  \FunctionTok{theme}\NormalTok{(}\AttributeTok{legend.position =} \StringTok{"bottom"}\NormalTok{)}
\end{Highlighting}
\end{Shaded}

\includegraphics{Intermediate-ggplot2_files/figure-latex/unnamed-chunk-9-1.pdf}

Add a line of best fit corresponding to a simple linear regression model
fit separately for males and females using geom\_smooth().

\begin{Shaded}
\begin{Highlighting}[]
\NormalTok{framinghamSub }\SpecialCharTok{|\textgreater{}}
  \FunctionTok{ggplot}\NormalTok{(}\FunctionTok{aes}\NormalTok{(}\AttributeTok{x =}\NormalTok{ SYSBP, }\AttributeTok{y =}\NormalTok{ DIABP,}
             \AttributeTok{size =}\NormalTok{ CIGPDAY,}
             \AttributeTok{color =}\NormalTok{ SexNew)) }\SpecialCharTok{+}
  \FunctionTok{geom\_point}\NormalTok{(}\AttributeTok{alpha =} \FloatTok{0.3}\NormalTok{) }\SpecialCharTok{+}
  \FunctionTok{geom\_smooth}\NormalTok{(}\AttributeTok{method =} \StringTok{"lm"}\NormalTok{, }\AttributeTok{se =} \ConstantTok{FALSE}\NormalTok{,}
              \AttributeTok{color =} \StringTok{"\#0076B6"}\NormalTok{, }\AttributeTok{size =} \FloatTok{0.8}\NormalTok{) }\SpecialCharTok{+}
  \FunctionTok{scale\_color\_colorblind}\NormalTok{() }\SpecialCharTok{+}
  \FunctionTok{facet\_grid}\NormalTok{(. }\SpecialCharTok{\textasciitilde{}}\NormalTok{ SexNew) }\SpecialCharTok{+}
  \FunctionTok{guides}\NormalTok{(}\AttributeTok{color =} \StringTok{"none"}\NormalTok{) }\SpecialCharTok{+}
  \FunctionTok{labs}\NormalTok{(}\AttributeTok{title =} \StringTok{"Diastolic by Systolic pressure"}\NormalTok{,}
       \AttributeTok{y =} \StringTok{"Diastolic BP (mmHg)"}\NormalTok{,}
       \AttributeTok{x =} \StringTok{"Systolic BP (mmHg)"}\NormalTok{,}
       \AttributeTok{caption =} \StringTok{"Data Source: Framington Heart Study, https://www.framinghamheartstudy.org/"}\NormalTok{,}
       \AttributeTok{size =} \StringTok{"Cigarettes Smoked daily"}\NormalTok{) }\SpecialCharTok{+}
  \FunctionTok{theme}\NormalTok{(}\AttributeTok{legend.position =} \StringTok{"bottom"}\NormalTok{)}
\end{Highlighting}
\end{Shaded}

\includegraphics{Intermediate-ggplot2_files/figure-latex/unnamed-chunk-10-1.pdf}

Add the estimated regression equations to each sub-plot using the ggpubr
package and adding a stat\_regline\_equation(label.x = 210, label.y =
40, size = 3.2) layer.

\begin{Shaded}
\begin{Highlighting}[]
\NormalTok{framinghamSub }\SpecialCharTok{|\textgreater{}}
  \FunctionTok{ggplot}\NormalTok{(}\FunctionTok{aes}\NormalTok{(}\AttributeTok{x =}\NormalTok{ SYSBP, }\AttributeTok{y =}\NormalTok{ DIABP,}
             \AttributeTok{size =}\NormalTok{ CIGPDAY,}
             \AttributeTok{color =}\NormalTok{ SexNew)) }\SpecialCharTok{+}
  \FunctionTok{geom\_point}\NormalTok{(}\AttributeTok{alpha =} \FloatTok{0.3}\NormalTok{) }\SpecialCharTok{+}
  \FunctionTok{geom\_smooth}\NormalTok{(}\AttributeTok{method =} \StringTok{"lm"}\NormalTok{, }\AttributeTok{se =} \ConstantTok{FALSE}\NormalTok{,}
              \AttributeTok{color =} \StringTok{"\#0076B6"}\NormalTok{, }\AttributeTok{size =} \FloatTok{0.8}\NormalTok{) }\SpecialCharTok{+}
  \FunctionTok{scale\_color\_colorblind}\NormalTok{() }\SpecialCharTok{+}
  \FunctionTok{facet\_grid}\NormalTok{(. }\SpecialCharTok{\textasciitilde{}}\NormalTok{ SexNew) }\SpecialCharTok{+}
  \FunctionTok{guides}\NormalTok{(}\AttributeTok{color =} \StringTok{"none"}\NormalTok{) }\SpecialCharTok{+}
  \FunctionTok{labs}\NormalTok{(}\AttributeTok{title =} \StringTok{"Diastolic by Systolic pressure"}\NormalTok{,}
       \AttributeTok{y =} \StringTok{"Diastolic BP (mmHg)"}\NormalTok{,}
       \AttributeTok{x =} \StringTok{"Systolic BP (mmHg)"}\NormalTok{,}
       \AttributeTok{caption =} \StringTok{"Data Source: Framington Heart Study \& the riskCommunicator package }\SpecialCharTok{\textbackslash{}n}\StringTok{ https://www.framinghamheartstudy.org/"}\NormalTok{,}
       \AttributeTok{size =} \StringTok{"Cigarettes Smoked daily"}\NormalTok{) }\SpecialCharTok{+}
  \FunctionTok{stat\_regline\_equation}\NormalTok{(}\AttributeTok{label.x =} \DecValTok{210}\NormalTok{,}\AttributeTok{label.y =} \DecValTok{40}\NormalTok{,}\AttributeTok{size =} \FloatTok{3.2}\NormalTok{) }\SpecialCharTok{+}
  \FunctionTok{theme}\NormalTok{(}\AttributeTok{legend.position =} \StringTok{"bottom"}\NormalTok{)}
\end{Highlighting}
\end{Shaded}

\includegraphics{Intermediate-ggplot2_files/figure-latex/unnamed-chunk-11-1.pdf}

\hypertarget{boxplots}{%
\subsection{Boxplots}\label{boxplots}}

Next, create a side-by-side box-plot where the y-axis is total
cholesterol (TOTCHOL) and the x-axis is current smoking status
(CURSMOKE). Increase the font size for the axes and title text in the
plot

\begin{Shaded}
\begin{Highlighting}[]
\CommentTok{\# creating a side{-}by{-}side box plots}
\NormalTok{framinghamSub }\SpecialCharTok{|\textgreater{}}
  \FunctionTok{ggplot}\NormalTok{(}\FunctionTok{aes}\NormalTok{(}\AttributeTok{x =}\NormalTok{ CurSmokeNew, }\AttributeTok{y =}\NormalTok{ TOTCHOL, }\AttributeTok{fill =}\NormalTok{ CurSmokeNew)) }\SpecialCharTok{+}
  \FunctionTok{geom\_boxplot}\NormalTok{() }\SpecialCharTok{+}
  \FunctionTok{scale\_fill\_manual}\NormalTok{(}\AttributeTok{values =} \FunctionTok{c}\NormalTok{(}\StringTok{"royalblue"}\NormalTok{,}\StringTok{"mediumseagreen"}\NormalTok{)) }\SpecialCharTok{+}
  \FunctionTok{labs}\NormalTok{(}\AttributeTok{title =} \StringTok{"Total Cholesterol by smoking status"}\NormalTok{,}
       \AttributeTok{y =} \StringTok{"Serum Total Cholesterol (mg/dL)"}\NormalTok{,}
       \AttributeTok{x =} \StringTok{"Current Smoker"}\NormalTok{,}
       \AttributeTok{caption =} \StringTok{"Data Source: Framington Heart Study \& the riskCommunicator package }\SpecialCharTok{\textbackslash{}n}\StringTok{ https://www.framinghamheartstudy.org/"}\NormalTok{) }\SpecialCharTok{+}
  \FunctionTok{theme\_few}\NormalTok{(}\AttributeTok{base\_size =} \DecValTok{16}\NormalTok{) }\SpecialCharTok{+}
  \FunctionTok{theme}\NormalTok{(}\AttributeTok{legend.position =} \StringTok{"none"}\NormalTok{)}
\end{Highlighting}
\end{Shaded}

\includegraphics{Intermediate-ggplot2_files/figure-latex/unnamed-chunk-12-1.pdf}

Color the boxes based on smoking status by manually specifying the
colors to be ``mediumseagreen'' and ``royalblue'', remove the legend,
and make the title and axis titles bold.

\begin{Shaded}
\begin{Highlighting}[]
\CommentTok{\# creating a side{-}by{-}side box plots}
\NormalTok{framinghamSub }\SpecialCharTok{|\textgreater{}}
  \FunctionTok{ggplot}\NormalTok{(}\FunctionTok{aes}\NormalTok{(}\AttributeTok{x =}\NormalTok{ CurSmokeNew, }\AttributeTok{y =}\NormalTok{ TOTCHOL, }\AttributeTok{fill =}\NormalTok{ CurSmokeNew)) }\SpecialCharTok{+}
  \FunctionTok{geom\_boxplot}\NormalTok{() }\SpecialCharTok{+}
  \FunctionTok{scale\_fill\_manual}\NormalTok{(}\AttributeTok{values =} \FunctionTok{c}\NormalTok{(}\StringTok{"royalblue"}\NormalTok{,}\StringTok{"mediumseagreen"}\NormalTok{)) }\SpecialCharTok{+}
  \FunctionTok{labs}\NormalTok{(}\AttributeTok{title =} \StringTok{"Total Cholesterol by smoking status"}\NormalTok{,}
       \AttributeTok{y =} \StringTok{"Serum Total Cholesterol (mg/dL)"}\NormalTok{,}
       \AttributeTok{x =} \StringTok{"Current Smoker"}\NormalTok{,}
       \AttributeTok{caption =} \StringTok{"Data Source: Framington Heart Study \& the riskCommunicator package }\SpecialCharTok{\textbackslash{}n}\StringTok{ https://www.framinghamheartstudy.org/"}\NormalTok{) }\SpecialCharTok{+}
  \FunctionTok{theme\_few}\NormalTok{(}\AttributeTok{base\_size =} \DecValTok{16}\NormalTok{) }\SpecialCharTok{+}
  \FunctionTok{theme}\NormalTok{(}\AttributeTok{legend.position =} \StringTok{"none"}\NormalTok{, }
        \AttributeTok{plot.title =} \FunctionTok{element\_text}\NormalTok{(}\AttributeTok{face =} \StringTok{"bold"}\NormalTok{, }\AttributeTok{color =} \StringTok{"red"}\NormalTok{),}
        \AttributeTok{axis.title.x =} \FunctionTok{element\_text}\NormalTok{(}\AttributeTok{face =} \StringTok{"bold"}\NormalTok{),}
        \AttributeTok{axis.title.y =} \FunctionTok{element\_text}\NormalTok{(}\AttributeTok{face =} \StringTok{"bold"}\NormalTok{))}
\end{Highlighting}
\end{Shaded}

\includegraphics{Intermediate-ggplot2_files/figure-latex/unnamed-chunk-13-1.pdf}

In a new plot, modify the side-by-side box-plots we created to be
faceted by the sex of the participant using the facet\_grid() function
and columns to break up the subplots.

\begin{Shaded}
\begin{Highlighting}[]
\CommentTok{\# creating a side{-}by{-}side box plots}
\NormalTok{framinghamSub }\SpecialCharTok{|\textgreater{}}
  \FunctionTok{ggplot}\NormalTok{(}\FunctionTok{aes}\NormalTok{(}\AttributeTok{x =}\NormalTok{ CurSmokeNew, }\AttributeTok{y =}\NormalTok{ TOTCHOL, }\AttributeTok{fill =}\NormalTok{ CurSmokeNew)) }\SpecialCharTok{+}
  \FunctionTok{geom\_boxplot}\NormalTok{() }\SpecialCharTok{+}
  \FunctionTok{facet\_grid}\NormalTok{(. }\SpecialCharTok{\textasciitilde{}}\NormalTok{ SexNew) }\SpecialCharTok{+}
  \FunctionTok{scale\_fill\_manual}\NormalTok{(}\AttributeTok{values =} \FunctionTok{c}\NormalTok{(}\StringTok{"royalblue"}\NormalTok{,}\StringTok{"mediumseagreen"}\NormalTok{)) }\SpecialCharTok{+}
  \FunctionTok{labs}\NormalTok{(}\AttributeTok{title =} \StringTok{"Total Cholesterol by smoking status"}\NormalTok{,}
       \AttributeTok{y =} \StringTok{"Serum Total Cholesterol (mg/dL)"}\NormalTok{,}
       \AttributeTok{x =} \StringTok{"Current Smoker"}\NormalTok{,}
       \AttributeTok{caption =} \StringTok{"Data Source: Framington Heart Study \& the riskCommunicator package }\SpecialCharTok{\textbackslash{}n}\StringTok{ https://www.framinghamheartstudy.org/"}\NormalTok{) }\SpecialCharTok{+}
  \FunctionTok{theme\_few}\NormalTok{(}\AttributeTok{base\_size =} \DecValTok{16}\NormalTok{) }\SpecialCharTok{+}
  \FunctionTok{theme}\NormalTok{(}\AttributeTok{legend.position =} \StringTok{"none"}\NormalTok{, }
        \AttributeTok{plot.title =} \FunctionTok{element\_text}\NormalTok{(}\AttributeTok{face =} \StringTok{"bold"}\NormalTok{, }\AttributeTok{color =} \StringTok{"red"}\NormalTok{),}
        \AttributeTok{axis.title.x =} \FunctionTok{element\_text}\NormalTok{(}\AttributeTok{face =} \StringTok{"bold"}\NormalTok{),}
        \AttributeTok{axis.title.y =} \FunctionTok{element\_text}\NormalTok{(}\AttributeTok{face =} \StringTok{"bold"}\NormalTok{))}
\end{Highlighting}
\end{Shaded}

\includegraphics{Intermediate-ggplot2_files/figure-latex/unnamed-chunk-14-1.pdf}

\hypertarget{linechart}{%
\subsection{Linechart}\label{linechart}}

Make a line graph that shows the average cigarettes per day (CIGPDAY) by
age (AGE), with separate lines by the sex of the participant (SEX).

\begin{Shaded}
\begin{Highlighting}[]
\NormalTok{framinghamSub }\SpecialCharTok{|\textgreater{}}
  \FunctionTok{ggplot}\NormalTok{() }\SpecialCharTok{+}
  \FunctionTok{stat\_summary}\NormalTok{(}\FunctionTok{aes}\NormalTok{(}\AttributeTok{x =}\NormalTok{ AGE,}
                   \AttributeTok{y =}\NormalTok{ CIGPDAY,}
                   \AttributeTok{color =}\NormalTok{ SexNew,}
                   \AttributeTok{group =}\NormalTok{ SexNew),}
               \AttributeTok{geom =} \StringTok{"line"}\NormalTok{,}
               \AttributeTok{fun.y =}\NormalTok{ mean,}
               \AttributeTok{size =} \DecValTok{1}\NormalTok{) }\SpecialCharTok{+}
  \FunctionTok{scale\_color\_colorblind}\NormalTok{() }\SpecialCharTok{+}
  \FunctionTok{labs}\NormalTok{(}\AttributeTok{title =} \StringTok{"Average cigarettes per day by age and sex"}\NormalTok{,}
       \AttributeTok{y =} \StringTok{"Average cigarettes per day"}\NormalTok{,}
       \AttributeTok{x =} \StringTok{"Age (years)"}\NormalTok{,}
       \AttributeTok{color =} \StringTok{"Sex"}\NormalTok{,}
       \AttributeTok{caption =} \StringTok{"Data Source: Framington Heart Study \& the riskCommunicator package }\SpecialCharTok{\textbackslash{}n}\StringTok{ https://www.framinghamheartstudy.org/"}\NormalTok{) }\SpecialCharTok{+}
  \FunctionTok{theme\_few}\NormalTok{(}\AttributeTok{base\_size =} \DecValTok{16}\NormalTok{) }\SpecialCharTok{+}
  \FunctionTok{theme}\NormalTok{(}\AttributeTok{plot.title =} \FunctionTok{element\_text}\NormalTok{(}\AttributeTok{face =} \StringTok{"bold"}\NormalTok{, }\AttributeTok{color =} \StringTok{"red"}\NormalTok{),}
        \AttributeTok{axis.title.x =} \FunctionTok{element\_text}\NormalTok{(}\AttributeTok{face =} \StringTok{"bold"}\NormalTok{),}
        \AttributeTok{axis.title.y =} \FunctionTok{element\_text}\NormalTok{(}\AttributeTok{face =} \StringTok{"bold"}\NormalTok{))}
\end{Highlighting}
\end{Shaded}

\includegraphics{Intermediate-ggplot2_files/figure-latex/unnamed-chunk-15-1.pdf}

\hypertarget{combining-plots-with-patchwork}{%
\subsection{Combining plots with
patchwork}\label{combining-plots-with-patchwork}}

Combine the line chart and the faceted scatter plots together into a
single graphic using the patchwork package, with 1 plot per row and the
line chart on top.

\begin{Shaded}
\begin{Highlighting}[]
\NormalTok{my\_linechart }\OtherTok{\textless{}{-}}\NormalTok{ framinghamSub }\SpecialCharTok{|\textgreater{}}
  \FunctionTok{ggplot}\NormalTok{() }\SpecialCharTok{+}
  \FunctionTok{stat\_summary}\NormalTok{(}\FunctionTok{aes}\NormalTok{(}\AttributeTok{x =}\NormalTok{ AGE,}
                   \AttributeTok{y =}\NormalTok{ CIGPDAY,}
                   \AttributeTok{color =}\NormalTok{ SexNew,}
                   \AttributeTok{group =}\NormalTok{ SexNew),}
               \AttributeTok{geom =} \StringTok{"line"}\NormalTok{,}
               \AttributeTok{fun.y =}\NormalTok{ mean,}
               \AttributeTok{size =} \DecValTok{1}\NormalTok{) }\SpecialCharTok{+}
  \FunctionTok{scale\_color\_colorblind}\NormalTok{() }\SpecialCharTok{+}
  \FunctionTok{labs}\NormalTok{(}\AttributeTok{title =} \StringTok{"Average cigarettes per day by age and sex"}\NormalTok{,}
       \AttributeTok{y =} \StringTok{"Average cigarettes per day"}\NormalTok{,}
       \AttributeTok{x =} \StringTok{"Age (years)"}\NormalTok{,}
       \AttributeTok{color =} \StringTok{"Sex"}\NormalTok{,}
       \AttributeTok{caption =} \StringTok{"Data Source: Framington Heart Study \& the riskCommunicator package }\SpecialCharTok{\textbackslash{}n}\StringTok{ https://www.framinghamheartstudy.org/"}\NormalTok{) }\SpecialCharTok{+}
  \FunctionTok{theme\_few}\NormalTok{(}\AttributeTok{base\_size =} \DecValTok{10}\NormalTok{) }\SpecialCharTok{+}
  \FunctionTok{theme}\NormalTok{(}\AttributeTok{plot.title =} \FunctionTok{element\_text}\NormalTok{(}\AttributeTok{face =} \StringTok{"bold"}\NormalTok{, }\AttributeTok{color =} \StringTok{"red"}\NormalTok{),}
        \AttributeTok{axis.title.x =} \FunctionTok{element\_text}\NormalTok{(}\AttributeTok{face =} \StringTok{"bold"}\NormalTok{),}
        \AttributeTok{axis.title.y =} \FunctionTok{element\_text}\NormalTok{(}\AttributeTok{face =} \StringTok{"bold"}\NormalTok{))}

\NormalTok{my\_scatterplot }\OtherTok{\textless{}{-}}\NormalTok{ framinghamSub }\SpecialCharTok{|\textgreater{}}
  \FunctionTok{ggplot}\NormalTok{(}\FunctionTok{aes}\NormalTok{(}\AttributeTok{x =}\NormalTok{ SYSBP, }\AttributeTok{y =}\NormalTok{ DIABP,}
             \AttributeTok{size =}\NormalTok{ CIGPDAY,}
             \AttributeTok{color =}\NormalTok{ SexNew)) }\SpecialCharTok{+}
  \FunctionTok{geom\_point}\NormalTok{(}\AttributeTok{alpha =} \FloatTok{0.3}\NormalTok{) }\SpecialCharTok{+}
  \FunctionTok{geom\_smooth}\NormalTok{(}\AttributeTok{method =} \StringTok{"lm"}\NormalTok{, }\AttributeTok{se =} \ConstantTok{FALSE}\NormalTok{,}
              \AttributeTok{color =} \StringTok{"\#0076B6"}\NormalTok{, }\AttributeTok{size =} \FloatTok{0.8}\NormalTok{) }\SpecialCharTok{+}
  \FunctionTok{scale\_color\_colorblind}\NormalTok{() }\SpecialCharTok{+}
  \FunctionTok{facet\_grid}\NormalTok{(. }\SpecialCharTok{\textasciitilde{}}\NormalTok{ SexNew) }\SpecialCharTok{+}
  \FunctionTok{guides}\NormalTok{(}\AttributeTok{color =} \StringTok{"none"}\NormalTok{) }\SpecialCharTok{+}
  \FunctionTok{labs}\NormalTok{(}\AttributeTok{title =} \StringTok{"Diastolic by Systolic pressure"}\NormalTok{,}
       \AttributeTok{y =} \StringTok{"Diastolic BP (mmHg)"}\NormalTok{,}
       \AttributeTok{x =} \StringTok{"Systolic BP (mmHg)"}\NormalTok{,}
       \AttributeTok{caption =} \StringTok{"Data Source: Framington Heart Study \& the riskCommunicator package }\SpecialCharTok{\textbackslash{}n}\StringTok{ https://www.framinghamheartstudy.org/"}\NormalTok{,}
       \AttributeTok{size =} \StringTok{"Cigarettes Smoked daily"}\NormalTok{) }\SpecialCharTok{+}
  \FunctionTok{stat\_regline\_equation}\NormalTok{(}\AttributeTok{label.x =} \DecValTok{210}\NormalTok{,}\AttributeTok{label.y =} \DecValTok{40}\NormalTok{,}\AttributeTok{size =} \FloatTok{3.2}\NormalTok{) }\SpecialCharTok{+}
  \FunctionTok{theme}\NormalTok{(}\AttributeTok{legend.position =} \StringTok{"bottom"}\NormalTok{)}

\FunctionTok{library}\NormalTok{(patchwork)}

\NormalTok{my\_linechart }\SpecialCharTok{/}\NormalTok{ my\_scatterplot}
\end{Highlighting}
\end{Shaded}

\includegraphics{Intermediate-ggplot2_files/figure-latex/unnamed-chunk-16-1.pdf}

\hypertarget{barchart}{%
\subsection{Barchart}\label{barchart}}

\begin{Shaded}
\begin{Highlighting}[]
\NormalTok{framinghamSub }\OtherTok{\textless{}{-}}\NormalTok{ framinghamSub }\SpecialCharTok{|\textgreater{}} 
  \FunctionTok{mutate}\NormalTok{(}\AttributeTok{CholesterolCat =} \FunctionTok{case\_when}\NormalTok{(TOTCHOL }\SpecialCharTok{\textless{}} \DecValTok{200} \SpecialCharTok{\textasciitilde{}} \StringTok{"Normal"}\NormalTok{,}
\NormalTok{                                    TOTCHOL }\SpecialCharTok{\textgreater{}=} \DecValTok{200} \SpecialCharTok{\&}\NormalTok{  TOTCHOL }\SpecialCharTok{\textless{}} \DecValTok{240} \SpecialCharTok{\textasciitilde{}} \StringTok{"Borderline high"}\NormalTok{,}
\NormalTok{                                    TOTCHOL }\SpecialCharTok{\textgreater{}} \DecValTok{240} \SpecialCharTok{\textasciitilde{}} \StringTok{"High"}\NormalTok{,}
                         \ConstantTok{TRUE} \SpecialCharTok{\textasciitilde{}} \FunctionTok{as.character}\NormalTok{(}\ConstantTok{NA}\NormalTok{)))}
\end{Highlighting}
\end{Shaded}

Create a bar chart displaying the number of participants falling in each
cholesterol category based on Johns Hopkins' definitions using
geom\_bar(). Also, remove people under 40 and those without recorded
cholesterol levels (missing values for CholesterolCat) from the plot by
using the code filter(AGE \textgreater= 40, !is.na(CholesterolCat)) when
piping the data into each subsequent ggplot() call.

Recreate the bar chart, this time reordering the categories to show
Normal, Borderline high, and then High from left to right using the
fct\_relevel() function.

Change the color of the inside of the bars based on the cholesterol
category using a color-blind friendly palette, make the outline of the
bars black in color, facet by the sex of the participant, and remove the
legend.

\begin{Shaded}
\begin{Highlighting}[]
\NormalTok{my\_bar }\OtherTok{\textless{}{-}}\NormalTok{ framinghamSub }\SpecialCharTok{|\textgreater{}} 
  \FunctionTok{filter}\NormalTok{(AGE }\SpecialCharTok{\textgreater{}=} \DecValTok{40}\NormalTok{, }\SpecialCharTok{!}\FunctionTok{is.na}\NormalTok{(CholesterolCat)) }\SpecialCharTok{|\textgreater{}}
  \FunctionTok{ggplot}\NormalTok{(}\FunctionTok{aes}\NormalTok{(}\AttributeTok{x =} \FunctionTok{fct\_relevel}\NormalTok{(CholesterolCat,}\StringTok{"Normal"}\NormalTok{,}\StringTok{"Borderline high"}\NormalTok{, }\StringTok{"High"}\NormalTok{),}
             \AttributeTok{fill =}\NormalTok{ CholesterolCat)) }\SpecialCharTok{+}
  \FunctionTok{scale\_fill\_viridis\_d}\NormalTok{() }\SpecialCharTok{+}
  \FunctionTok{geom\_bar}\NormalTok{(}\AttributeTok{color =} \StringTok{"black"}\NormalTok{) }\SpecialCharTok{+} 
  \FunctionTok{facet\_grid}\NormalTok{(.}\SpecialCharTok{\textasciitilde{}}\NormalTok{SexNew) }\SpecialCharTok{+}
  \FunctionTok{labs}\NormalTok{(}\AttributeTok{title =} \StringTok{"Distribution of Cholesterol levels"}\NormalTok{,}
       \AttributeTok{x =} \StringTok{"Cholesterol category"}\NormalTok{,}
       \AttributeTok{y =} \StringTok{"Frequency"}\NormalTok{,}
       \AttributeTok{caption =} \StringTok{"Data Source: Framington Heart Study \& the riskCommunicator package }\SpecialCharTok{\textbackslash{}n}\StringTok{ https://www.framinghamheartstudy.org/"}\NormalTok{) }\SpecialCharTok{+}
  \FunctionTok{theme}\NormalTok{(}\AttributeTok{legend.position =} \StringTok{"none"}\NormalTok{)}

\NormalTok{my\_bar}
\end{Highlighting}
\end{Shaded}

\includegraphics{Intermediate-ggplot2_files/figure-latex/unnamed-chunk-18-1.pdf}

Remove the extra space between the bars and the horizontal axis.

\begin{Shaded}
\begin{Highlighting}[]
\NormalTok{my\_bar }\OtherTok{\textless{}{-}}\NormalTok{ my\_bar }\SpecialCharTok{+}
  \FunctionTok{scale\_y\_continuous}\NormalTok{(}\AttributeTok{expand =} \FunctionTok{expansion}\NormalTok{(}\AttributeTok{mult =} \FunctionTok{c}\NormalTok{(}\DecValTok{0}\NormalTok{,}\FloatTok{0.1}\NormalTok{)))}

\NormalTok{my\_bar}
\end{Highlighting}
\end{Shaded}

\includegraphics{Intermediate-ggplot2_files/figure-latex/unnamed-chunk-19-1.pdf}

Lastly, use the coor\_flip() function to turn the bar chart into a
horizontal bar chart instead.

\begin{Shaded}
\begin{Highlighting}[]
\NormalTok{my\_bar }\SpecialCharTok{+}
  \FunctionTok{coord\_flip}\NormalTok{()}
\end{Highlighting}
\end{Shaded}

\includegraphics{Intermediate-ggplot2_files/figure-latex/unnamed-chunk-20-1.pdf}

Recreate the line chart below showing the median BMI by age for smokers
and non-smokers. Hint: use a scale\_color\_manual() layer to match the
colors.

\begin{Shaded}
\begin{Highlighting}[]
\NormalTok{framinghamSub }\SpecialCharTok{|\textgreater{}}
  \FunctionTok{ggplot}\NormalTok{() }\SpecialCharTok{+}
  \FunctionTok{stat\_summary}\NormalTok{(}\FunctionTok{aes}\NormalTok{(}\AttributeTok{x =}\NormalTok{ AGE,}
                   \AttributeTok{y =}\NormalTok{ BMI,}
                   \AttributeTok{color =}\NormalTok{ CurSmokeNew,}
                   \AttributeTok{group =}\NormalTok{ CurSmokeNew),}
               \AttributeTok{geom =} \StringTok{"line"}\NormalTok{,}
               \AttributeTok{fun.y =}\NormalTok{ median,}
               \AttributeTok{size =} \DecValTok{1}\NormalTok{) }\SpecialCharTok{+}
  \FunctionTok{scale\_color\_manual}\NormalTok{(}\AttributeTok{values =} \FunctionTok{c}\NormalTok{(}\StringTok{"mediumseagreen"}\NormalTok{,}\StringTok{"royalblue"}\NormalTok{)) }\SpecialCharTok{+}
  \FunctionTok{labs}\NormalTok{(}\AttributeTok{title =} \StringTok{"Median BMI by age and smoking status"}\NormalTok{,}
       \AttributeTok{y =} \StringTok{"Median Body Mass Index"}\NormalTok{,}
       \AttributeTok{x =} \StringTok{"Age (years)"}\NormalTok{,}
       \AttributeTok{color =} \StringTok{"Current Smoker"}\NormalTok{,}
       \AttributeTok{caption =} \StringTok{"Data Source: Framington Heart Study \& the riskCommunicator package }\SpecialCharTok{\textbackslash{}n}\StringTok{ https://www.framinghamheartstudy.org/"}\NormalTok{) }\SpecialCharTok{+}
  \FunctionTok{theme\_few}\NormalTok{(}\AttributeTok{base\_size =} \DecValTok{16}\NormalTok{) }\SpecialCharTok{+}
  \FunctionTok{theme}\NormalTok{(}\AttributeTok{legend.position =} \StringTok{"bottom"}\NormalTok{) }\SpecialCharTok{+}
  \FunctionTok{xlim}\NormalTok{(}\DecValTok{40}\NormalTok{,}\DecValTok{70}\NormalTok{)}
\end{Highlighting}
\end{Shaded}

\includegraphics{Intermediate-ggplot2_files/figure-latex/unnamed-chunk-21-1.pdf}

We can use scale\_x\_continuous(limits = c(40,70)) or xlim(40,70) or
dplyr::filter(between(AGE.40,70))

Recreate the faceted dumbbell chart below showing the same information
as the faceted bar chart. Hint: One way to create a dumbbell chart is to
use geom\_point(size = 4, pch = 21) and geom\_line().

\begin{Shaded}
\begin{Highlighting}[]
\NormalTok{framinghamSub }\SpecialCharTok{|\textgreater{}} 
\NormalTok{  dplyr}\SpecialCharTok{::}\FunctionTok{count}\NormalTok{(CholesterolCat,SexNew,}\AttributeTok{.drop =} \ConstantTok{FALSE}\NormalTok{) }\SpecialCharTok{|\textgreater{}}
\NormalTok{  dplyr}\SpecialCharTok{::}\FunctionTok{filter}\NormalTok{( }\SpecialCharTok{!}\FunctionTok{is.na}\NormalTok{(CholesterolCat)) }\SpecialCharTok{|\textgreater{}}
  \FunctionTok{ggplot}\NormalTok{(}\FunctionTok{aes}\NormalTok{(}\AttributeTok{x =}\NormalTok{ n,}\AttributeTok{y =}\NormalTok{ SexNew, }\AttributeTok{color =}\NormalTok{ CholesterolCat, }\AttributeTok{fill =}\NormalTok{ CholesterolCat)) }\SpecialCharTok{+} 
  \FunctionTok{scale\_fill\_viridis\_d}\NormalTok{() }\SpecialCharTok{+}
  \FunctionTok{geom\_line}\NormalTok{(}\FunctionTok{aes}\NormalTok{(}\AttributeTok{group =}\NormalTok{ SexNew),}\AttributeTok{color =} \StringTok{"black"}\NormalTok{) }\SpecialCharTok{+}
  \FunctionTok{geom\_point}\NormalTok{(}\AttributeTok{color =} \StringTok{"black"}\NormalTok{,}\AttributeTok{size =} \DecValTok{4}\NormalTok{, }\AttributeTok{pch =} \DecValTok{21}\NormalTok{) }\SpecialCharTok{+}
  \FunctionTok{labs}\NormalTok{(}\AttributeTok{title =} \StringTok{"Distribution of Cholesterol levels"}\NormalTok{,}
       \AttributeTok{x =} \StringTok{"Frequency"}\NormalTok{,}
       \AttributeTok{y =} \StringTok{""}\NormalTok{,}
       \AttributeTok{color =} \StringTok{"Cholesterol category:"}\NormalTok{,}
       \AttributeTok{caption =} \StringTok{"Data Source: Framington Heart Study \& the riskCommunicator package }\SpecialCharTok{\textbackslash{}n}\StringTok{ https://www.framinghamheartstudy.org/"}\NormalTok{) }\SpecialCharTok{+}
  \FunctionTok{theme}\NormalTok{(}\AttributeTok{legend.position =} \StringTok{"bottom"}\NormalTok{)}
\end{Highlighting}
\end{Shaded}

\includegraphics{Intermediate-ggplot2_files/figure-latex/unnamed-chunk-22-1.pdf}

\end{document}
